\section{Results}
\label{sec:results}

We first look at the runtime differences of the different approaches based on different starting points, under the datasets briefly mentioned in Section \ref{sec:method} and detailed in Appendices \ref{ecoli} and \ref{kapferer}. Each subsection to follow will also contain the parameter estimates and a measure of the difference between the method and starting point combinations.

\subsection{E. Coli (no self-loop)}

\paragraph{Equilibrium Expectation (EE)}

\begin{table}[H]
 \centering
 \begin{tabular}{||c | c | c | c | c||} 
 \hline
 Starting Point & User & System & Elapsed \\
 \hline\hline
 zeros & 21.052 & 0.025 & 21.099 \\ 
 \hline
 zeros and edges & 6.763 & 0.004 & 6.777 \\
 \hline
 MPLE & 17.863 & 0.027 & 17.897 \\
 \hline
 \end{tabular}
 \label{t:ecoli1_ee}
 \caption{E. Coli dataset run-times with various starting point configurations using EE as the estimation method}
\end{table}

\begin{table}[H]
\centering
\begin{tabular}{|| c | c | c | c | c | c | c ||}
\hline
Starting point & Edges & Degree2 & Degree3 & Degree4 & Degree5 & Gwdeg.fixed.0.25 \\
\hline
Zeros & -5.737519 & -3.323081 & -4.305612 & -3.518558 & -2.636180 & 2.9922375 \\
\hline
Zeros and edges & -4.539951 &-2.110865 & -3.098017 & -3.209092 & -3.681314 & 1.7391186 \\
\hline
MPLE & 15.534803 & 32.362031 & 32.199131 & 20.771601 & 12.224798 & -196.0182076 \\
\hline
\end{tabular}
\label{t:params_ecoli_ee}
\caption{E. Coli parameter estimates with various starting points using EE as the estimation method}
\end{table}

\paragraph{Stochastic Approximation (SA)}

\begin{table}[H]
\centering
\begin{tabular}{||c | c | c | c | c||}
 \hline
 Starting Point & User & System & Elapsed \\ 
 \hline
 zeros & 21.134 & 0.087 & 21.795 \\
 \hline
 zeros and edges & 896.263 & 0.264 & 897.086 \\ 
 \hline
 MPLE & 17.572 & 0.027 & 17.610 \\
 \hline
 \end{tabular}
 \label{t:ecoli1_sa}
 \caption{E. Coli dataset run-times with various starting point configurations using SA as the estimation method}
 \end{table}

\begin{table}[H]
\centering
\begin{tabular}{|| c | c | c | c | c | c | c ||}
\hline
Starting point & Edges & Degree2 & Degree3 & Degree4 & Degree5 & Gwdeg.fixed.0.25 \\
\hline
Zeros & -5.737519 & -3.323081 & -4.305612 & -3.518558 & -2.636180 & 2.9922375 \\
\hline
Zeros and edges & -4.539951 &-2.110865 & -3.098017 & -3.209092 & -3.681314 & 1.7391186 \\
\hline
MPLE & 15.534803 & 32.362031 & 32.199131 & 20.771601 & 12.224798 & -196.0182076 \\
\hline
\end{tabular}
\label{t:params_ecoli_sa}
\caption{E. Coli parameter estimates with various starting points using SA as the estimation method}
\end{table}

\paragraph{Markov Chain Maximum Likelihood Estimator (MCMLE)}

\begin{table}[H]
\centering
 \begin{tabular}{||c | c | c | c | c||} 
 \hline
 Starting Point & User & System & Elapsed \\ 
 \hline
 zeros & 301.087 & 3.712 & 373.466 \\
 \hline
 zeros and edges & 46.978 & 3.267 & 373.466 \\ 
 \hline
 MPLE & 43.854 & 3.185 & 125.712 \\ 
 \hline
\end{tabular}
\label{t:ecoli1_mcmle}
\caption{E. Coli dataset run-times with various starting point configurations using MCMLE as the estimation method}
\end{table}

\begin{table}[H]
\centering
\begin{tabular}{|| c | c | c | c | c | c | c ||}
\hline
Starting point & Edges & Degree2 & Degree3 & Degree4 & Degree5 & Gwdeg.fixed.0.25 \\
\hline
Zeros & -1.680262 & 497.776421 & 189.629113 & 130.370015 & 35.555459 & -446.7001373 \\
\hline
Zeros and edges & -5.111108 & -1.402672 & -2.505505 & -2.271429 & -3.267121 & 1.5884273 \\
\hline
MPLE & -4.981500 & -1.397047 & -2.246059 & -2.009961 & -1.587440 & 0.8127727 \\
\hline
\end{tabular}
\label{t:params_ecoli_mcmle}
\caption{E. Coli parameter estimates with various starting points using MCMLE as the estimation method}
\end{table}

\subsection{E. Coli (self-loop)}

\paragraph{Equilibrium Expectation (EE)}

\begin{table}[H]
\centering
 \begin{tabular}{||c | c | c | c | c||} 
 \hline
 Method & Starting Point & User & System & Elapsed \\
 \hline\hline
 EE & zeros & 22.007	 & 0.034 & 22.041 \\ 
 \hline
 EE & zeros and edges & 8.160 & 0.002 & 8.155 \\
 \hline
 EE & MPLE & 19.046 & 0.043 & 19.098 \\
 \hline
 \end{tabular}
 \label{t:ecoli2_ee}
 \caption{E. Coli dataset with self-loop run times with various starting point configurations and EE as the estimation method}
\end{table}


\begin{table}[H]
\centering
\scriptsize
\begin{tabular}{|| c | c | c | c | c | c | c | c | c ||}
\hline
Starting point & Edges & Degree2 & Degree3 & Degree4 & Degree5 & Gwdeg0.25 & Mix False & Mix True \\
\hline
Zeros & -10.716486 & -6.594619 & -6.689404 & -7.215315 & -4.544823 & 10.6444020 & 11.29079006 & 35.7647417 \\
\hline
Zeros and edges & -5.852937 & -1.927183 &-2.109867 & -2.128186 & -2.384952 & 2.4482905 & 1.83567510 & 1.8338598 \\
\hline
MPLE & 2.224877	& 14.058709 & 13.586779 & 8.812083 & 5.728130 & -82.8397531 & -1.85029310 & -4.1102634 \\
\hline
\end{tabular}
\label{t2:params_ecoli_ee}
\caption{E. Coli parameter estimates with various starting points using EE as the estimation method}
\end{table}


\paragraph{Stochastic Approximation (SA)}

\begin{table}[H]
\centering
\begin{tabular}{||c | c | c | c | c||}
 \hline
 SA & zeros & 22.128 & 0.113 & 22.460 \\
 \hline
 SA & zeros and edges & 932.005 & 0.493 & 1215.180 \\ 
 \hline
 SA & MPLE & 18.799 & 0.094 & 18.933 \\ 
 \hline
 \end{tabular}
 \label{t:ecoli2_sa}
 \caption{E. Coli dataset with self-loop run times with various starting point configurations and SA as the estimation method}
\end{table}


\begin{table}[H]
\scriptsize
\centering
\begin{tabular}{|| c | c | c | c | c | c | c | c | c ||}
\hline
Starting point & Edges & Degree2 & Degree3 & Degree4 & Degree5 & Gwdeg0.25 & Mix False & Mix True \\
\hline
Zeros & -6.243631 & -5.642640 & -5.536328 & -4.552708 & -1.396533 & 0.6543856 & 10.07940661 & 1.4683291 \\
\hline
Zeros and edges & -5.816048 & -1.309330 & -2.025997 & -1.777584 & -2.264008 & 2.0434754 & 1.55321296 & 1.1751994 \\
\hline
MPLE & 3.190518	& 12.771154 & 9.827869 & 8.806080 & 1.497652 & -77.5304319 & -1.15712833 &-7.7680711 \\
\hline
\end{tabular}
\label{t2:params_ecoli_sa}
\caption{E. Coli parameter estimates with various starting points using SA as the estimation method}
\end{table}

\paragraph{Markov Chain Maximum Likelihood Estimator (MCMLE)}

\begin{table}[H]
\centering
\begin{tabular}{||c | c | c | c | c||}
 \hline
 MCMLE & zeros & 361.232 & 2.940 & 366.182 \\  
 \hline
 MCMLE & zeros and edges & 32.287 & 1.438 & 33.682 \\ 
 \hline
 MCMLE & MPLE & 22.193 & 0.835 & 22.816 \\  
 \hline
\end{tabular}
\label{t:ecoli2}
\caption{E. Coli dataset with self-loop run times with various starting point configurations and MCMLE as the estimation method}
\end{table}

\begin{table}[H]
\centering
\scriptsize
\begin{tabular}{|| c | c | c | c | c | c | c | c | c ||}
\hline
Starting point & Edges & Degree2 & Degree3 & Degree4 & Degree5 & Gwdeg0.25 & Mix False & Mix True \\
\hline
Zeros & -1.643217 & 591.371338 & 225.284319 & 154.882969 & 42.240810 & -530.6913835 & 0.07455501 & -0.2704657 \\
\hline
Zeros and edges & -5.825112 & -1.361900 & -2.035242 & -1.781577 & -2.323788 & 2.3079946 & 1.53632447 & 1.1967021 \\
\hline
MPLE & -5.849533	 & -1.356081 & -2.012842 & -1.755260 & -2.330215 & 2.3329175 & 1.56200515 & 1.2046862 \\
\hline
\end{tabular}
\label{t2:params_ecoli_mcmle}
\caption{E. Coli parameter estimates with various starting points using MCMLE as the estimation method}
\end{table}


\subsection{Kapferer (no gwdsp)}


\subsection{Kapferer (self-loop)}


