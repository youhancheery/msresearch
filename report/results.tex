\section{Results}
\label{sec:results}

We first look at the runtime differences of the different approaches based on different starting points, under the datasets briefly mentioned in Section \ref{sec:method} and detailed in Appendices \ref{ecoli} and \ref{kapferer}. Each subsection to follow will also contain the parameter estimates and a measure of the difference between the method and starting point combinations.

\subsection{Model Set 1: E. Coli With No Mixing}


\paragraph{Equilibrium Expectation (EE)}

\begin{table}[H]
 \centering
 \begin{tabular}{||c | c | c | c | c||} 
 \hline
 Starting Point & User & System & Elapsed \\
 \hline\hline
 zeros & 21.052 & 0.025 & 21.099 \\ 
 \hline
 zeros and edges & 6.763 & 0.004 & 6.777 \\
 \hline
 MPLE & 17.863 & 0.027 & 17.897 \\
 \hline
 \end{tabular}
 \label{t:ecoli1_ee}
 \caption{E. Coli dataset run-times with various starting point configurations using EE as the estimation method}
\end{table}

\begin{table}[H]
\centering
\scriptsize
\begin{tabular}{|| c | c | c | c | c | c | c ||}
\hline
Starting point & Edges & Degree2 & Degree3 & Degree4 & Degree5 & Gwdeg.fixed.0.25 \\
\hline
Zeros & -5.737519 & -3.323081 & -4.305612 & -3.518558 & -2.636180 & 2.9922375 \\
\hline
Zeros and edges & -4.539951 &-2.110865 & -3.098017 & -3.209092 & -3.681314 & 1.7391186 \\
\hline
MPLE & 15.534803 & 32.362031 & 32.199131 & 20.771601 & 12.224798 & -196.0182076 \\
\hline
\end{tabular}
\label{t:params_ecoli_ee}
\caption{E. Coli parameter estimates with various starting points using EE as the estimation method}
\end{table}

\paragraph{Stochastic Approximation (SA)}

\begin{table}[H]
\centering
\begin{tabular}{||c | c | c | c | c||}
 \hline
 Starting Point & User & System & Elapsed \\ 
 \hline
 zeros & 21.134 & 0.087 & 21.795 \\
 \hline
 zeros and edges & 896.263 & 0.264 & 897.086 \\ 
 \hline
 MPLE & 17.572 & 0.027 & 17.610 \\
 \hline
 \end{tabular}
 \label{t:ecoli1_sa}
 \caption{E. Coli dataset run-times with various starting point configurations using SA as the estimation method}
 \end{table}

\begin{table}[H]
\centering
\scriptsize
\begin{tabular}{|| c | c | c | c | c | c | c ||}
\hline
Starting point & Edges & Degree2 & Degree3 & Degree4 & Degree5 & Gwdeg.fixed.0.25 \\
\hline
Zeros & -5.737519 & -3.323081 & -4.305612 & -3.518558 & -2.636180 & 2.9922375 \\
\hline
Zeros and edges & -4.539951 &-2.110865 & -3.098017 & -3.209092 & -3.681314 & 1.7391186 \\
\hline
MPLE & 15.534803 & 32.362031 & 32.199131 & 20.771601 & 12.224798 & -196.0182076 \\
\hline
\end{tabular}
\label{t:params_ecoli_sa}
\caption{E. Coli parameter estimates with various starting points using SA as the estimation method}
\end{table}

\paragraph{Markov Chain Maximum Likelihood Estimator (MCMLE)}

\begin{table}[H]
\centering
 \begin{tabular}{||c | c | c | c | c||} 
 \hline
 Starting Point & User & System & Elapsed \\ 
 \hline
 zeros & 301.087 & 3.712 & 373.466 \\
 \hline
 zeros and edges & 46.978 & 3.267 & 373.466 \\ 
 \hline
 MPLE & 43.854 & 3.185 & 125.712 \\ 
 \hline
\end{tabular}
\label{t:ecoli1_mcmle}
\caption{E. Coli dataset run-times with various starting point configurations using MCMLE as the estimation method}
\end{table}

\begin{table}[H]
\centering
\scriptsize
\begin{tabular}{|| c | c | c | c | c | c | c ||}
\hline
Starting point & Edges & Degree2 & Degree3 & Degree4 & Degree5 & Gwdeg.fixed.0.25 \\
\hline
Zeros & -1.680262 & 497.776421 & 189.629113 & 130.370015 & 35.555459 & -446.7001373 \\
\hline
Zeros and edges & -5.111108 & -1.402672 & -2.505505 & -2.271429 & -3.267121 & 1.5884273 \\
\hline
MPLE & -4.981500 & -1.397047 & -2.246059 & -2.009961 & -1.587440 & 0.8127727 \\
\hline
\end{tabular}
\label{t:params_ecoli_mcmle}
\caption{E. Coli parameter estimates with various starting points using MCMLE as the estimation method}
\end{table}

\subsection{Model Set 2: E. Coli With Self Mixing}

\paragraph{Equilibrium Expectation (EE)}

\begin{table}[H]
\centering
 \begin{tabular}{|| c | c | c | c||} 
 \hline
 Starting Point & User & System & Elapsed \\
 \hline\hline
 Zeros & 22.007	 & 0.034 & 22.041 \\ 
 \hline
 Zeros and edges & 8.160 & 0.002 & 8.155 \\
 \hline
 MPLE & 19.046 & 0.043 & 19.098 \\
 \hline
 \end{tabular}
 \label{t:ecoli2_ee}
 \caption{E. Coli dataset with self-loop run times with various starting point configurations and EE as the estimation method}
\end{table}


\begin{table}[H]
\centering
\scriptsize
\begin{tabular}{|| c | c | c | c | c | c | c | c | c ||}
\hline
Starting point & Edges & Degree2 & Degree3 & Degree4 & Degree5 & Gwdeg0.25 & Mix False & Mix True \\
\hline
Zeros & -10.716486 & -6.594619 & -6.689404 & -7.215315 & -4.544823 & 10.6444020 & 11.29079006 & 35.7647417 \\
\hline
Zeros and edges & -5.852937 & -1.927183 &-2.109867 & -2.128186 & -2.384952 & 2.4482905 & 1.83567510 & 1.8338598 \\
\hline
MPLE & 2.224877	& 14.058709 & 13.586779 & 8.812083 & 5.728130 & -82.8397531 & -1.85029310 & -4.1102634 \\
\hline
\end{tabular}
\label{t2:params_ecoli_ee}
\caption{E. Coli parameter estimates with various starting points using EE as the estimation method}
\end{table}


\paragraph{Stochastic Approximation (SA)}

\begin{table}[H]
\centering
\begin{tabular}{|| c | c | c | c||}
 \hline
 Starting Point & User & System & Elapsed \\
 \hline
 Zeros & 22.128 & 0.113 & 22.460 \\
 \hline
 Zeros and edges & 932.005 & 0.493 & 1215.180 \\ 
 \hline
 MPLE & 18.799 & 0.094 & 18.933 \\ 
 \hline
 \end{tabular}
 \label{t:ecoli2_sa}
 \caption{E. Coli dataset with self-loop run times with various starting point configurations and SA as the estimation method}
\end{table}


\begin{table}[H]
\scriptsize
\centering
\begin{tabular}{|| c | c | c | c | c | c | c | c | c ||}
\hline
Starting point & Edges & Degree2 & Degree3 & Degree4 & Degree5 & Gwdeg0.25 & Mix False & Mix True \\
\hline
Zeros & -6.243631 & -5.642640 & -5.536328 & -4.552708 & -1.396533 & 0.6543856 & 10.07940661 & 1.4683291 \\
\hline
Zeros and edges & -5.816048 & -1.309330 & -2.025997 & -1.777584 & -2.264008 & 2.0434754 & 1.55321296 & 1.1751994 \\
\hline
MPLE & 3.190518	& 12.771154 & 9.827869 & 8.806080 & 1.497652 & -77.5304319 & -1.15712833 &-7.7680711 \\
\hline
\end{tabular}
\label{t2:params_ecoli_sa}
\caption{E. Coli parameter estimates with various starting points using SA as the estimation method}
\end{table}

\paragraph{Markov Chain Maximum Likelihood Estimator (MCMLE)}

\begin{table}[H]
\centering
\begin{tabular}{|| c | c | c | c||}
 \hline
 Starting Point & User & System & Elapsed \\ 
 \hline
 Zeros & 361.232 & 2.940 & 366.182 \\  
 \hline
 Zeros and edges & 32.287 & 1.438 & 33.682 \\ 
 \hline
 MPLE & 22.193 & 0.835 & 22.816 \\  
 \hline
\end{tabular}
\label{t:ecoli2}
\caption{E. Coli dataset with self-loop run times with various starting point configurations and MCMLE as the estimation method}
\end{table}

\begin{table}[H]
\centering
\scriptsize
\begin{tabular}{|| c | c | c | c | c | c | c | c | c ||}
\hline
Starting point & Edges & Degree2 & Degree3 & Degree4 & Degree5 & Gwdeg0.25 & Mix False & Mix True \\
\hline
Zeros & -1.643217 & 591.371338 & 225.284319 & 154.882969 & 42.240810 & -530.6913835 & 0.07455501 & -0.2704657 \\
\hline
Zeros and edges & -5.825112 & -1.361900 & -2.035242 & -1.781577 & -2.323788 & 2.3079946 & 1.53632447 & 1.1967021 \\
\hline
MPLE & -5.849533	 & -1.356081 & -2.012842 & -1.755260 & -2.330215 & 2.3329175 & 1.56200515 & 1.2046862 \\
\hline
\end{tabular}
\label{t2:params_ecoli_mcmle}
\caption{E. Coli parameter estimates with various starting points using MCMLE as the estimation method}
\end{table}


\subsection{Model Set 3: Kapferer With No GWDSP Parameter}

\paragraph{Equilibrium Expectation}

\begin{table}[H]
\centering
\begin{tabular}{||c|c|c|c||}
\hline
Starting Point & User & System & Elapsed \\
\hline
Zeros & 147.615	& 0.075	& 147.761 \\
\hline
Zeros and edges & 38.984 & 0.009 & 39.002 \\
\hline
MPLE & 134.161 & 0.064 & 134.255 \\
\hline
\end{tabular}
\label{t:runtimes_kap1_ee}
\caption{Runtime for Equilibrium Expectation on Kapferer dataset using model 3 across the three tested starting points}
\end{table}

% parameters

\paragraph{Stochastic Approximation}

\begin{table}[H]
\centering
\begin{tabular}{||c|c|c|c||}
\hline
Starting Point & User & System & Elapsed \\
\hline
Zeros & 23.540 & 0.010 & 23.585 \\
\hline
Zeros and edges & 1114.139 & 0.259 & 1114.803 \\
\hline
MPLE & 21.153 & 0.009 & 21.193 \\
\hline
\end{tabular}
\label{t:runtimes_kap1_sa}
\caption{Runtime for Stochastic Approximation on Kapferer dataset using model 3 across the three tested starting points}
\end{table}

% parameters

\begin{table}[H]
\centering
\begin{tabular}{||c|c|c|c||}
  \hline
  Starting Point & Edges & gwesp.fixed.0.25 & gwdsp.fixed.0.25 \\ 
  \hline
  Zeros & -1.82 & 0.74 & -0.19 \\ 
  \hline
  Zeros and edges & -1.06 & 0.74 & -0.37 \\
  \hline
  MPLE & -2.88 & 1.38 & -0.14 \\
  \hline
\end{tabular}
\label{t:params_kap1_ee}
\caption{Parameters of Kapferer dataset under model 3 with Equilibrium Expectation}
\end{table}


\subsection{Model Set 4: Kapferer With GWDSP Parameter}

\paragraph{Equilibrium Expectation}

\begin{table}[H]
\centering
\begin{tabular}{||c|c|c|c||}
\hline
Starting Point & User & System & Elapsed \\
\hline
Zeros & 61.388 & 0.026 & 61.437 \\
\hline
Zeros and edges & 19.057 & 0.004 & 19.069 \\
\hline
MPLE & 56.211 & 0.030 & 56.253 \\
\hline 
\end{tabular}
\label{t:runtimes_kap2_ee}
\caption{Runtimes of model set 3 on Kapferer data, with EE as the estimation method across different starting points}
\end{table}

% params

\begin{table}[]
\centering
\begin{tabular}{||c|c|c|c|c||}
  \hline
 Starting Point & edges & gwdeg.fixed.0.25 & gwesp.fixed.0.25 & gwdsp.fixed.0.25 \\ 
  \hline
  Zeros & -4.29 & -9.74 & 3.41 & -1.75 \\
  \hline 
  Zeros and edges & -1.05 & -45.09 & 0.60 & -0.28 \\ 
  \hline
  MPLE & -2.95 & -0.17 & 1.39 & -0.12 \\ 
  \hline
\end{tabular}
\label{t:params_kap2_ee}
\caption{Parameters of model set 3 on Kapferer data, with EE as the estimation method across various starting configurations}
\end{table}

\paragraph{Stochastic Approximation}

\begin{table}[H]
\centering
\begin{tabular}{||c|c|c|c||}
\hline
Starting Point & User & System & Elapsed \\
\hline
Zeros & 61.248 & 0.036 & 61.586 \\
\hline
Zeros and edges & 2945.969 & 1.102 & 2950.486 \\
\hline 
MPLE & 56.096 & 0.042 & 56.158 \\
\hline 
\end{tabular}
\label{t:runtimes_kap2_sa}
\caption{Runtimes of model set 4 on Kapferer data, with EE as the estimation method of choice across the three different starting points}
\end{table}

% params
\begin{table}[H]
\centering
\begin{tabular}{||c|c|c|c|c||}
  \hline
  Starting Point & edges & gwdeg.fixed.0.25 & gwesp.fixed.0.25 & gwdsp.fixed.0.25 \\ 
  \hline
  Zeros & -6.99 & -24.31 & 2.26 & 0.43 \\ 
  \hline
  Zeros and edges & -2.42 & 0.09 & 1.06 & -0.15 \\ 
  \hline
  MPLE & -2.94 & 0.12 & 1.38 & -0.12 \\ 
  \hline
\end{tabular}
\label{t:params_kap2_sa}
\caption{Parameters of model set 4 on Kapferer data, with SA as the estimation method across various starting configurations}
\end{table}

\subsection{Discussion}

First, we will consider the parameter estimates across the models in reference to the standard set by \cite{hummels2012} re-keyed in the table \ref{t:hummel_ecoli} below. We note that the parameter values in Hummel's paper were calculated through the stepping algorithm with the default ERGM starting point, the MPLE. Thus, the most comparative analysis can be made when considering the implementations above with respect to an MPLE starting point, though it is still worth exploring the variances in parameter estimates and differences in runtimes across all three models.

\begin{table}[H]
\centering
\begin{tabular}{||c|c|c||}
\hline
Parameter & Model 1 & Model 1 plus self-edges \\
\hline
Edges & -5.07 (0.027, 0.012) & -5.83 (0.067, 0.013) \\
\hline 
Degree-2 & -1.47 (0.126, 0.025) & -1.36 (0.141, 0.014) \\
\hline
Degree-3 & -2.36 (0.192, 0.027) & -2.03 (0.206, 0.010) \\ 
\hline
Degree-4 & -2.03 (0.206, 0.020) & -1.79 (0.239, 0.009) \\
\hline
Degree-5 & -2.91 (0.415, 0.183) & -2.32 (0.409, 0.035) \\
\hline
GWDEG(0.25) & 1.86 (0.174, 0.094) & 2.32 (0.336, 0.050) \\
\hline
Node + Self & N/A & 1.55 (0.037, 0.0002) \\
\hline
Self + Node & N/A & 1.21 (0.118, 0.002) \\
\hline
\end{tabular}
\label{t:hummel_ecoli}
\caption{Hummel parameter estimates for the E. Coli models (defined ``model set 1'' and ``model set 2'' above) obtained using Stepping (\cite{hummels2012})}
\end{table} 

We see that the equilibrium expectation algorithm in ``Model set 1'' (so a comparison with ``model 1'' in table \ref{t:hummel_ecoli}) produces noticeably different results to the parameters shown here. Similarly, equating equilibrium expectation with ``model 1 plus self-edges'' 


\begin{table}[H]
\centering
\begin{tabular}{||c|c|c||}
\hline
Parameter & Model 1 & Model 2 \\
\hline
Edges & -3.016163 & -3.17598 \\
\hline
GWESP(0.25) & 1.444937 & 1.53531 \\
\hline
GWDSP(0.25) & 0.322320 & -0.11757 \\
\hline
GWDegree(0.25) & NA & 0.43127 \\
\hline
\end{tabular}
\label{t:hummel_kapferer}
\caption{Parameter estimation of the Kapferer dataset with equivalent models in \cite{hummels2012}}
\end{table}

