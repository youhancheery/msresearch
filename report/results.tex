\section{Results}
\label{section:results}

We first look at the runtime differences of the different approaches based on different starting points, under the datasets briefly mentioned in Section \ref{section:method} and detailed in Appendices \ref{ecoli} and \ref{kapferer}. Each subsection to follow will also contain the parameter estimates and a measure of the difference between the method and starting point combinations.

\subsection{E. Coli (no self-loop)}

\paragraph{Equilibrium Expectation (EE)}

\begin{table}[!ht]
 \centering
 \begin{tabular}{||c | c | c | c | c||} 
 \hline
 Starting Point & User & System & Elapsed \\
 \hline\hline
 zeros & 21.052 & 0.025 & 21.099 \\ 
 \hline
 zeros and edges & 6.763 & 0.004 & 6.777 \\
 \hline
 MPLE & 17.863 & 0.027 & 17.897 \\
 \hline
 \end{tabular}
 \label{t:ecoli1_ee}
 \caption{E. Coli dataset run-times with various starting point configurations using EE as the estimation method}
\end{table}

\paragraph{Stochastic Approximation (SA)}

\begin{table}[!ht]
\centering
\begin{tabular}{||c | c | c | c | c||}
 \hline
 Starting Point & User & System & Elapsed \\ 
 \hline
 zeros & 21.134 & 0.087 & 21.795 \\
 \hline
 zeros and edges & 896.263 & 0.264 & 897.086 \\ 
 \hline
 MPLE & 17.572 & 0.027 & 17.610 \\
 \hline
 \end{tabular}
 \label{t:ecoli1_sa}
 \caption{E. Coli dataset run-times with various starting point configurations using SA as the estimation method}
 \end{table}

\paragraph{Markov Chain Maximum Likelihood Estimator (MCMLE)}

\begin{table}[!ht]
\centering
 \begin{tabular}{||c | c | c | c | c||} 
 \hline
 Starting Point & User & System & Elapsed \\ 
 \hline
 zeros & 301.087 & 3.712 & 373.466 \\
 \hline
 zeros and edges & 46.978 & 3.267 & 373.466 \\ 
 \hline
 MPLE & 43.854 & 3.185 & 125.712 \\ 
 \hline
\end{tabular}
\label{t:ecoli1_mcmle}
\caption{E. Coli dataset run-times with various starting point configurations using MCMLE as the estimation method}
\end{table}

\subsection{E. Coli (self-loop)}

\paragraph{Equilibrium Expectation (EE)}

\begin{table}[!ht]
\centering
 \begin{tabular}{||c | c | c | c | c||} 
 \hline
 Method & Starting Point & User & System & Elapsed \\
 \hline\hline
 EE & zeros & 22.007	 & 0.034 & 22.041 \\ 
 \hline
 EE & zeros and edges & 8.160 & 0.002 & 8.155 \\
 \hline
 EE & MPLE & 19.046 & 0.043 & 19.098 \\
 \hline
 \end{tabular}
 \label{t:ecoli2_ee}
 \caption{E. Coli dataset with self-loop run times with various starting point configurations and EE as the estimation method}
\end{table}

\paragraph{Stochastic Approximation (SA)}

\begin{table}[!ht]
\centering
\begin{tabular}{||c | c | c | c | c||}
 \hline
 SA & zeros & 22.128 & 0.113 & 22.460 \\
 \hline
 SA & zeros and edges & 932.005 & 0.493 & 1215.180 \\ 
 \hline
 SA & MPLE & 18.799 & 0.094 & 18.933 \\ 
 \hline
 \end{tabular}
 \label{t:ecoli2_sa}
 \caption{E. Coli dataset with self-loop run times with various starting point configurations and SA as the estimation method}
\end{table}

\paragraph{Markov Chain Maximum Likelihood Estimator (MCMLE)}

\begin{table}[!ht]
\centering
\begin{tabular}{||c | c | c | c | c||}
 \hline
 MCMLE & zeros & 361.232 & 2.940 & 366.182 \\  
 \hline
 MCMLE & zeros and edges & 32.287 & 1.438 & 33.682 \\ 
 \hline
 MCMLE & MPLE & 22.193 & 0.835 & 22.816 \\  
 \hline
\end{tabular}
\label{t:ecoli2}
\caption{E. Coli dataset with self-loop run times with various starting point configurations and MCMLE as the estimation method}
\end{table}

\subsection{Kapferer (no gwdsp)}

%\begin{center}
%\begin{table}
% \begin{tabular}{||c c c c c c||} 
% \hline
% Method & Starting Point & User & System & Elapsed \\ [0.5ex] 
% \hline\hline
% EE & zeros & 21.052 & 0.025 & 21.099 \\ 
% \hline
% EE & zeros but with edge parameter pre-estimated & 6.763 & 0.004 & 6.777 \\
% \hline
% EE & MPLE & 17.863 & 0.027 & 17.897 \\
% \hline
% SA & zeros & 21.134 & 0.087 & 21.795 \\
% \hline
% SA & zeros but with edge parameter pre-estimated & 896.263 & 0.264 & 897.086 \\ [1ex] 
% \hline
% SA & MPLE & 17.572 & 0.027 & 17.610 \\ [1ex] 
% \hline
% MCMLE & zeros & 301.087 & 3.712 & 373.466 \\ [1ex] 
% \hline
% MCMLE & zeros but with edge parameter pre-estimated & 46.978 & 3.267 & 373.466 \\ [1ex] 
% \hline
% MCMLE & MPLE & 43.854 & 3.185 & 125.712 \\ [1ex] 
% \hline
%\end{tabular}
%\label{t:kap1}
%\caption{Kapferer dataset run-times with various starting point configurations}
%\end{table}
%\end{center}


\subsection{Kapferer (self-loop)}



%\begin{center}
%\begin{table}
% \begin{tabular}{||c c c c c c||} 
% \hline
% Method & Starting Point & User & System & Elapsed \\ [0.5ex] 
% \hline\hline
% EE & zeros & 21.052 & 0.025 & 21.099 \\ 
% \hline
% EE & zeros but with edge parameter pre-estimated & 6.763 & 0.004 & 6.777 \\
% \hline
% EE & MPLE & 17.863 & 0.027 & 17.897 \\
% \hline
% SA & zeros & 21.134 & 0.087 & 21.795 \\
% \hline
% SA & zeros but with edge parameter pre-estimated & 896.263 & 0.264 & 897.086 \\ [1ex] 
% \hline
% SA & MPLE & 17.572 & 0.027 & 17.610 \\ [1ex] 
% \hline
% MCMLE & zeros & 301.087 & 3.712 & 373.466 \\ [1ex] 
% \hline
% MCMLE & zeros but with edge parameter pre-estimated & 46.978 & 3.267 & 373.466 \\ [1ex] 
% \hline
% MCMLE & MPLE & 43.854 & 3.185 & 125.712 \\ [1ex] 
% \hline
%\end{tabular}
%\label{t:kap2}
%\caption{Kapferer dataset with self-loop run-times with various starting point configurations}
%\end{table}
%\end{center}

