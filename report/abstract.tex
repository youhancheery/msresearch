\section{Abstract}
%%
Exponential Random Graph Models (ERGMs) are a family of network models that are useful in modelling complex network structures, such as corporate connections and social relations. ERGMs benefit from many of the properties that make their namesake, the exponential family, such a powerful device in the statistical toolkit.

Given the nature and complexity of modelling network interactions both in terms of statistical methodology and computational execution - many methods have been devised to estimate parameters of a network model accurately and within computational constraints. This challenge is due to the normalising constant that is core to the definition of an ERGM, often making parameter estimation intractable and difficult to compute. As such, initial methods approximated the likelihood in order to estimate parameters, an approach known as Maximum Pseudo Likelihood Estimation (MPLE) \cite{straussikeda1990}, while more current approaches leverage improvements in computation power and aim to reach the MLE via the Markov Chain Monte Carlo (MCMC). Benchmarking these MCMC methods will be the crux of this paper.

Within the class of MCMC estimation methods, a recent approach, Equilibrium Expectation (EE), proposes an efficient means to compute the parameters of large networks, a typically challenging computational problem \cite{eqexpectation}. By using Markov Chain properties at equilibrium it is able to make infrequent, albeit complex updates to the parameter values to reduce computation effort required. It follows on from methods popularised by Robbins and Monro \cite{robbinsmonro1951} and Stochastic Approximation \cite{snijders2002} which have been staples in the estimation of ERGM parameters for a number of years. This paper will look to benchmark the Equilibrium Expectation algorithm against these two methods by comparing properties across some of the more challenging network datasets covered in \cite{hummels2012}.

The next section introduces ERGMs and the challenge at hand, followed by a literature review spanning Robbins and Monro, to Stochastic Approximation by Snijder, followed on by a summary of Equilibrium Expectation. This is then followed by the methodology and results obtained in the analysis. 
