\hspace{0pt}
\vfill
\paragraph{Abstract}

Exponential Random Graph Models (ERGMs) are a family of network models that are useful in modelling complex network structures, such as corporate connections and social relations. ERGMs benefit from many of the properties that make their namesake, the exponential family, such a powerful device in statistical literature and practice. However, despite the benefits from being exponential family models, it is computationally challenging to estimate the parameters that define an ERGM model. Several methods have been devised to estimate parameters of a network model accurately and within computational constraints which will be explored in this work.

The challenge of parameter estimation is largely due to the normalising constant that is core to the definition of an ERGM often making computation intractable. As such, initial methods approximated the likelihood in order to estimate parameters, an approach known as Maximum Pseudo Likelihood Estimation (MPLE) (\cite{straussikeda1990}), while more current approaches leverage improvements in computation power and aim to reach the maximum likelihood estimate using Monte Carlo simulation. 

Within the class of Monte Carlo estimation methods, a recent approach, Equilibrium Expectation (EE) (\cite{eqexpectation}), proposes an a fast and scalable means to compute the parameters of large networks, a typically challenging computational problem. By using Markov Chain properties at equilibrium it is able to make infrequent, albeit complex updates to the parameter values to reduce computation effort required. It follows on from the works of Robbins and Monro and Snijders in the form of Stochastic Approximation, as well as Geyer and Thompson and Hunter and Handock respectively in the mode of Markov Chain Maximum Likelihood Estimation (MCMLE) which have been staples in the estimation of ERGM parameters for a number of years. This paper will look to benchmark the Equilibrium Expectation algorithm against these two methods by comparing properties across some of the more challenging network datasets covered in \cite{hummels2012}, namely the famous open source network datasets ``E. Coli'' and ``Kapferer''. 

This report is structured as follows: the succeeding section introduces ERGMs and the challenge at hand in more detail, followed by a literature review spanning Stochastic Approximation estimation by Snijder as an extension to the Robbins-Monro algorithm, Markov Chain Monte Carlo, and lastly a summary of the Equilibrium Expectation method. Post-literature review this report will then dive into both the method and the results of the analysis in sections \ref{sec:method} and \ref{sec:results} respectively.

\vfill
\hspace{0pt}
