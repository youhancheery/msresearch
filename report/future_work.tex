\section{Future Work}
\label{sec:futurework}

A number of improvements can be done to build on this analysis in future, listed below. We categorise these in terms of singularly improving this analysis, as opposed to improving the theme of the analysis.

\begin{itemize}
\item While we only considered starting point variation and gain manipulation in this paper, there's an opportunity to continue experimenting with other hyper-parameters of the Equilibrium Expectation algorithm.
\item Due to time constraints, we were unable to complete benchmarking of the MCMLE approach with EE and SA for the Kapferer data. Convergence was not achieved on the models.
\item Running the models in parallel to show more realistic times in the context of modern computational statistics. 
\item Experimenting with different root finding methods in the Equilibrium Expectation algorithm to test the runtime. 
\item Increasing the number of datasets we test on, particularly focusing on a larger variance in network sizes. Looking at much smaller networks as well as much larger networks than those used in this paper.
\item Benchmarking with Contrastive Divergence as a starting point for the models (\cite{krivitsky2017}), and considering other estimation algorithms such as partial stepping. 
\end{itemize}